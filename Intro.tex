%%%%%%%%%%%%%%%%%%%%%%%%%%%%%%
\chapter*{Introduction}
%%%%%%%%%%%%%%%%%%%%%%%%%%%%%%

Le métabolisme correspond à l'ensemble des processus complexes et incessants de transformation de matière et d'énergie par la cellule ou l'organisme, au cours des phénomènes d'édification et de dégradation organiques (anabolisme et catabolisme). 
Ces molécules, appelées aussi métabolites, et les réactions dans lesquelles ils interviennent forment des réseaux métaboliques. \\

Analyser les réseaux métaboliques peut parfois s'avérer complexe étant donnée l'importance de la taille de certains. L'outil bioinformatique devient donc vite indispensable dans le traitement de telles données.\\ 
Le but de notre projet sera donc de mettre en place une interface graphique pour le logiciel \emph{regEfmtool}, qui est un outil de calcul des modes élémentaires de flux.
Un mode élémentaire est un chemin métabolique au sein d'un réseau, c'est à dire un jeu de réactions uniques pour lequel, à l'état stationnaire, le flux global à travers le processus est nul.\\
La création de cette interface aura pour but de rendre l'utilisation de \emph{regEfmtool} plus conviviale et de faire en sorte que les résultats générés soient visuellement interprétables. 
