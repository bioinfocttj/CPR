%%%%%%%%%%%%%%%%%%%%%%%%%%%%%%
\chapter{Choix et justifications}
%%%%%%%%%%%%%%%%%%%%%%%%%%%%%%
Le choix de langages impératifs orientés web pour cette interface a été effectué pour des raisons personnelles et par pertinence quand au projet demandé.\\
Tout dabord le choix de logiciels web pour la création de l'interface de \emph{regEfmtool} à été éffectué dans le but d'apprendre de nouveaux langages (pour la majorité des personnes composant ce groupe de projet).\\
Ensuite, le choix d'une interface web permet à tout utilisateur, quel que soit son système d'exploitation, de pouvoir utiliser le logiciel. En effet, \emph{regEfmtool} 
étant installé et exécuté sur une machine \emph{UNIX} qui est reliée à d'autres machines sur un réseau local (ou sur internet) 
n'importe quelle autre machine peut avoir accès au logiciel via sont navigateur internet.\\
Cela permet notament a des machines même peu puissantes de pouvoir effectuer des simulations conséquentes si le logiciel est installé sur un serveur ou un ordinateur performant.\\
Dans le cas d'une installation sur une machine non reliée à un réseau de partage de \emph{regEfmtool} l'execution est néammoins possible.
