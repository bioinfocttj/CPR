%%%%%%%%%%%%%%%%%%%%%%%%%%%%%%
\chapter{Contexte}
%%%%%%%%%%%%%%%%%%%%%%%%%%%%%%
Le métabolisme d'une cellule est un système complexe de transformations moléculaires et énergétiques qui se déroulent 
de manière ininterrompue dans la cellule et mettant en jeu un ensemble de réactions dites métaboliques. 
Ces réactions impliques différents types de métabolites qui, suivant leurs positions dans la réaction, sont appelées substrats 
ou produits et sont généralement catalysées par des enzymes.

Afin de faciliter leurs études, on associe un ensemble de ces réactions métaboliques de façon à représenter les grandes 
fonctions métaboliques (glycolyse, etc) créant ainsi des réseaux métaboliques. Cette modélisation de réactions permet 
de réaliser des requêtes complexes comme, par exemple, le calcul (et la prédiction) de tous les métabolites pouvant 
être générés à partir d’un ensemble de composés sources.

Quelques logiciels disponibles internationalement permettent de travailler et d'automatiser l'étude de ces réseaux et proposent 
des outils tels que le calcul des modes élémentaires de flux ou la recherche de minimal \textit{cut sets}. 
Cependant, ils sont soit dépendants de logiciels non libre comme MATLAB, soit ne possèdent pas d'interface utilisateur conviviale.\\
