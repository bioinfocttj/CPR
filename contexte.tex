%%%%%%%%%%%%%%%%%%%%%%%%%%%%%%
\chapter{Contexte}
%%%%%%%%%%%%%%%%%%%%%%%%%%%%%%
\section{Sujet}
Le métabolisme d'une cellule est un système complexe de transformations moléculaires et énergétiques qui se déroulent 
de manière ininterrompue dans la cellule et mettant en jeu un ensemble de réactions dites métaboliques. 
Ces réactions impliques différents types de métabolites qui, suivant leurs positions dans la réaction, sont appelées 
substrats ou produits et sont généralement catalysées par des enzymes.

Afin de faciliter leurs études, on associe un ensemble de ces réactions métaboliques de façon à représenter les grandes 
fonctions métaboliques (glycolyse, etc) créant ainsi des réseaux métaboliques. Cette modélisation de réactions permet 
de réaliser des requêtes complexes comme, par exemple, le calcul (et la prédiction) de tous les métabolites pouvant 
être générés à partir d’un ensemble de composés sources.

Quelques logiciels disponibles internationalement permettent de travailler et d'automatiser l'étude de ces réseaux 
et proposent des outils tels que le calcul des modes élémentaires de flux ou la recherche de minimal \textit{cut sets}. 
Cependant, ils sont soit dépendants de logiciels non libre comme MATLAB, soit ne possèdent pas d'interface utilisateur 
conviviale tel que RegEFMTool, logiciel sur lequel nous nous appuierons pour ce projet, qui ne s'utilise qu'à partir 
d'une console.\\

\section{Objectif}
Dans le cadre de l'ue CPR(\textbf{a modifier}) il nous est demandé de réaliser un projet caractérisé par la création
d'un programme (package et documentation à fournir à la fin du projet) nous permettant d'approfondir ou apprendre 
un langage de programmation et de réaliser une analyse critique du travail effectué.
L'objectif de ce projet sera donc de réaliser une interface graphique du programme RegEfmtool qui ne peut être utilisé, 
à l'heure actuelle, que par des commandes à l'aide d'une console.

Afin de réaliser cette interface graphique, nous nous appuierons sur les technologies web. En effet, beaucoup de 
programmes utilisent une interface de type site web leur permettant d'une part de créer une interface graphique 
conviviale (à partir du langage html, feuilles de style (css), etc) tout en permettant une certaine modularité 
(possible grace au langage JavaScript et php). 