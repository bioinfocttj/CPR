%%%%%%%%%%%%%%%%%%%%%%%%%%%%%%
\chapter{Besoins fonctionnels et non fonctionnels}
%%%%%%%%%%%%%%%%%%%%%%%%%%%%%%

\section{Besoins fonctionnels}

\subsection{Interphace Homme-Machine (IHM)}
L'interface web que nous allons créer devra permettre de charger la description d'un réseau déjà créé ou d'en créer un nouveau, mais également de modifier les descriptions de ce réseau.\\
Elle donnera également le moyen à l'utilisateur de saisir les fonctions et optios qui l'intéressent puis de lancer les calculs des modes élémentaires de flux du réseau d'intérêt. De plus un scripte récurent de commandes pourra être enregistré et chargé par la suite.\\
Selon les caractéristiques du fichier utilisateur rentré, une liste des choix possibles pourra être proposé. Ainsi les résultats d'expériences similaires doivent pouvoir être comparés.
Enfin, la visualisation des résultats devra apparaître de façon claire à l'utilisateur à travers un fichier généré ou sur l'nterface web.

\subsection{Aide en ligne des fonctions et options}
Une aide en ligne sera engendré lors du passage du curseur de la souris sur la fonction ou l'option de son choix. Ainsi l'utilisateur pourra avoir plus d'informations sur la commande concernée.



\section{Besoins non fonctionnels}

\subsection{Portabilité}
L'utilisation de regEfmtool s'appuie sur des logiciels existants, tel que le langage de programmation Java 1.7. Ceux-ci devront être libre d'utilisation pour le secteur académique et délivrés avec tous les fichiers de configurations des outils écrits ou liés au programme. Les programme que nous écrirons devront etre indépendant du système d'eploitation ou fonctionner sous Linux comme regEfmtool.

\subsection{Parer les erreurs techniques}
L'interface devra permettre à l'utilisateur d'etre informer en cas d'erreur lors de la charge d'un fichier non compatible avec regEfmtool ou contenant des erreurs d'écritures ou lorsque les paramètres entrés ne sont pas en accords avec la fonction ou l'option sélectionnée. 

\subsection{Documentations}
L'écriture du code sera constitué de commentaire qui permettrons la maintenance du code ainsi qu'une éventuelle amélioration de ce dernier par un tier.
L'interface web créée, quand à elle, devra être fournie avec une documentation sur son installation, son utilisation, sa maintenance et une charte graphique déclarant les différents attributs du site (couleurs utilisées, police, logo, image,...).

\subsection{Dates}
Début du projet mi-octobre 2012. Remise du cahier des charges et présentation du sujet le 28 novembre 2012.  Mi-décembre 2012: point sur l'avancement du projet.Rendez-vous pour l'avancement du code en janvier 2013 Remise du projet fin février 2013.