%%%%%%%%%%%%%%%%%%%%%%%%%%%%%%
\chapter{Besoins fonctionnels et non fonctionnels}
%%%%%%%%%%%%%%%%%%%%%%%%%%%%%%

\section{Besoins fonctionnels}

L'interface web que nous allons créer devra permettre de charger la description d'un réseau déjà créé ou d'en créer un nouveau, mais également de modifier les descriptions de ce réseau.\\
Elle donnera également le moyen à l'utilisateur de saisir les paramètres qui l'intéressent puis de lancer les calculs des modes élémentaires de flux du réseau d'intérêt.\\
Selon les caractéristiques du fichier utilisateur rentré une liste des choix possibles pourra être proposé. Les résultats d'expériences similaires doivent pouvoir être comparés.
Enfin, la visualisation des résultats devra apparaître de façon claire à l'utilisateur.


\section{Besoins non fonctionnels}

L'interface web créée devra être fournie avec une documentation.\\
Si elle s'appuie sur des logiciels existants, ceux-ci devront être libre d'utilisation pour le secteur académique.