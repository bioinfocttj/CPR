%%%%%%%%%%%%%%%%%%%%%%%%%%%%%%
\chapter{Choix et justifications}
%%%%%%%%%%%%%%%%%%%%%%%%%%%%%%

\section{Langages}
Nous avons fait le choix de langages impératifs orientés Web pour cette interface pour des raisons personnelles et par pertinence quand au projet demandé. De plus, le choix de créer un logiciel Web pour la création de l'interface de \emph{regEfmtool} va nous permettre d'apprendre de nouveaux langages (pour la majorité des personnes composant ce groupe de projet). \\
Il a été décidé d'utiliser les langages dédiés aux technologies Web, tels que HTML 5\footnote{HTML: Hypertext Markup Language.} pour l'interface graphique couplé aux feuilles de styles CSS\footnote{Cascading Style Sheets.}, JavaScript et PHP pour le fonctionnement du site Web, ainsi que la persistance des données pour les fonctions d'analyse et de comparaison de réseaux. D'autres langages pourront être utilisés au fur et à mesure de l'avancement s'ils apparaissent être plus adaptés que ceux précédemment cité.

\section{Accessibilité}
Pour permettre un accès facile à notre programme depuis n'importe quel ordinateur, indépendamment du système d'exploitation et de l'endroit, nous avons décidé de développer notre projet sous la forme d'une application Web. En effet, une connexion à la page Web du site suffira pour avoir accès à toute ses fonctionnalités. \emph{RegEfmtool} est installé et exécuté sur une machine \emph{UNIX}, qui est reliée à d'autres machines sur un réseau local (ou sur internet). De ce fait, n'importe quelle autre machine peut avoir accès au logiciel via son navigateur internet.\\
Cela permet notamment à des machines, même peu puissantes, de pouvoir effectuer des simulations conséquentes si le logiciel est installé sur un serveur ou un ordinateur performant.\\
Dans le cas d'une installation sur une machine non reliée à un réseau de partage de \emph{regEfmtool} l'exécution est néanmoins possible.

%\section{Base de données}
%Nous allons utiliser la base de données du KEGG afin de permettre une visualisation des réseaux métaboliques. C'est un moyen plus convivial d'observation des résultats qu'une simple liste de réactions et qu'une série de matrices. 
