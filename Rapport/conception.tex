%%%%%%%%%%%%%%%%%%%%%%%%%%%%%%
\chapter{Conception}
%%%%%%%%%%%%%%%%%%%%%%%%%%%%%%

\section{Règles générales}
Lorsque l'utilisateur a choisi de créer un nouveau réseau métabolique, il doit créer les règles qui permettront de le définir. Cette étape n'est possible que lorsque l'utilisateur a créer son réseau et donné les réactions réversibles.\\

Tout d'abord, l'utilisateur choisi le nombre de réactions qui composent la règle qu'il veut créer.
Une fois ce choix effectué, il clique sur le bouton "ok". S'affichent alors deux menus déroulant par réaction. Le premier permet de choisir la réaction et le second sa valeur (0 ou 1).\\

L'utilisateur ne peut sélectionner le nom de la deuxième réaction que lorsqu'il a choisi la première, ce qui l'empêche de choisir deux fois la même réaction. Le même principe s'applique a chaque nouvelle réaction choisie.\\

Enfin, quand l'utilisateur a sélectionné toutes les réactions ainsi que leur valeur, il clique sur le bouton "Ajouter", la règle est alors écrite dans le fichier \emph{grfile}.