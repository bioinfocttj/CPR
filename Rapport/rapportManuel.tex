\documentclass[12pt,a4paper]{report}
\usepackage[utf8]{inputenc}
\usepackage[francais]{babel}
\usepackage[T1]{fontenc}
\usepackage{amsmath}
\usepackage{amsfonts}
\usepackage{amssymb}
\usepackage{url}
\usepackage{ucs}
\usepackage{graphicx}
\usepackage{lmodern}
\usepackage{xcolor}
\usepackage[left=2cm,right=2cm,top=2cm,bottom=2cm]{geometry}
\author{Typhaine PL}
\setcounter{tocdepth}{1} 
\usepackage{lmodern}
\usepackage{fancyvrb}
\definecolor{Zgris}{rgb}{0.87,0.85,0.85}

% Pour avoir du code coloré
\usepackage{listings}
\lstset{
language=HTML,        % choix du langage
basicstyle=\ttfamily\footnotesize,       % taille de la police du code
keywordstyle=\ttfamily\bfseries\color{red},  %style des mots clefs
commentstyle=\ttfamily\slshape\color{darkgreen},  %style des commentaires
identifierstyle=  \ttfamily\color{blue},  %style des identificateurs
stringstyle= \ttfamily\color{green},  %style des chaines de caractères
showstringspaces=false,
numbers=left,                   % placer les numéros de lignes à gauche (left)
numberstyle=\footnotesize,        % taille de la police des numéros
numbersep=6pt,            % distance entre le code et sa numérotation
tabsize=4,
xleftmargin=17pt,
escapeinside={\%*}{*)},
breaklines=true,
breakatwhitespace=true,
string=[b]{'}
}

\newsavebox{\BBbox}
\newenvironment{DDbox}[1]{
\begin{lrbox}{\BBbox}\begin{minipage}{\linewidth}}
{\end{minipage}\end{lrbox}\noindent\colorbox{Zgris}{\usebox{\BBbox}} \\
[.5cm]}

\makeatletter
\def\clap#1{\hbox to 0pt{\hss #1\hss}}%
\def\ligne#1{%
\hbox to \hsize{%
\vbox{\centering #1}}}%
\def\haut#1#2#3{%
\hbox to \hsize{%
\rlap{\vtop{\raggedright #1}}%
\hss
\clap{\vtop{\centering #2}}%
\hss
\llap{\vtop{\raggedleft #3}}}}%
\def\bas#1#2#3{%
\hbox to \hsize{%
\rlap{\vbox{\raggedright #1}}%
\hss
\clap{\vbox{\centering #2}}%
\hss
\llap{\vbox{\raggedleft #3}}}}%
\def\maketitle{%
\thispagestyle{empty}\vbox to \vsize{%
\haut{}{\@blurb}{}
\vfill
\vspace{1cm}
\begin{flushleft}
\usefont{OT1}{ptm}{m}{n}
\huge \@title
\end{flushleft}
\par
\hrule height 4pt
\par
\begin{flushright}
\usefont{OT1}{phv}{m}{n}
\Large \@author
\par
\end{flushright}
\vspace{1cm}
\vfill
\begin{center}
\includegraphics[width=0.3\textwidth]{../Images/logo/logo.png}\\
\vspace{2cm}
\includegraphics[width=0.1\textwidth]{../Images/logounibdx.png}
\end{center}
\vfill
\bas{}{\@date}{}
}%
\cleardoublepage
}
\def\date#1{\def\@date{#1}}
\def\author#1{\def\@author{#1}}
\def\title#1{\def\@title{#1}}
\def\location#1{\def\@location{#1}}
\def\blurb#1{\def\@blurb{#1}}
\date{\today}
\author{}
\title{}
\location{}\blurb{}
\makeatother

\title{Manuel de Maintenance de WRET}
\author{\textsc{FR\`ECHE} Arnaud, \textsc{H\'ERIC\'E} Charlotte, \textsc{MOLA} Saraï,\\ \textsc{PAYSAN-LAFOSSE} Typhaine, \textsc{SANSEN} Joris}
\location{Bordeaux}
\blurb{%
Universités de Bordeaux\\
Master 2 BioInformatique
\textbf{}\\[1em]
\textsc{Marie \textsc{Beurton-Aimar}}
}%
\begin{document}
\tableofcontents

\chapter{Architecture de WRET}
\section{Dossier et fichier}

Notre programme se trouve dans un dossier général « CPR/ ». Il est agencé dans deux dossier distinct: \\
\begin{itemize}
\item WRET
\item regEfmTool
\end{itemize} 

Le dossier WRET est composé de l'ensemble des fichiers constituant le site web ainsi que de dossiers contenant les langues et les images de WRET.\\

Voici le listing de l'ensemble des fichiers organisé selon leur extension :
\begin{itemize}
\item fichiers PHP :
\begin{itemize}
\item all\_results.php
\item copyrights.php
\item create.php
\item createFiles.php
\item createGrfile.php
\item display\_reults.php
\item file\_choose.php
\item finish\_files.php
\item generules.php
\item help.php
\item help\_create.php
\item help\_rules.php
\item index.php
\item init\_files.php
\item instal.php
\item load.php
\item load\_files.php
\item modifier.php
\item modifyRules.php
\item options.php
\item parse\_results.php
\item parser\_enzymes.php
\item parser\_metabolite.php
\item parser\_stoechiometry.php
\item reac.php
\item recupEnzymes.php
\item results.php
\end{itemize}
\item fichiers JavasScript :
\begin{itemize}
\item actionLoad.js
\item choixOptions.js
\item generules.js 
\end{itemize}
\item fichier Python :
\begin{itemize}
\item parser\_stoechiometry.py
\end{itemize}
\item fichier CSS :
\begin{itemize}
\item style.css
\end{itemize}
\end{itemize} 

Le dossier regEfmTool possède l'ensemble du logiciel \emph{RegEfmTool}
qu'il faut installer. Pour cela se référer au manuel d'installation.

\section{Modifications du code}
\subsection{Dépendances} 
Se référer au manuel d’installation.

\subsection{Définition des répertoires de stockage}

Par défaut, tous les répertoires sont définis par rapport au répertoire de WRET.


\subsection{Evolution du code}
\subsubsection{Nouveau fichier HTML}

Pour ajouter une page HTML il suffit de la crée et d'ajouter, sur un page existante, un lien vers cette nouvelle page.

Il y a possibilité d'ajouter un onglet au menu. Dans ce cas sur chaque page existante, dans le header, les références de l'onglet doivent \^etre ajoutées dans la div "menu" sur le modèle de l'exemple suivant : <li><a href="load.php">	<?php echo TXT_MENU_MODIFY; ?>	</a></li>.


\end{document}