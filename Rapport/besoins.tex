%%%%%%%%%%%%%%%%%%%%%%%%%%%%%%
\section{Analyse des besoins}
%%%%%%%%%%%%%%%%%%%%%%%%%%%%%%

\subsection{Besoins fonctionnels}

\subsubsection{Interface Homme Machine (IHM)}
L'interface Web que nous avons réalisé permet de créer un nouveau réseau métabolique pour \textit{regEfmTool} ou pour \emph{METATOOL}.\\
Elle donne également le moyen à l'utilisateur de saisir les fonctions et options qui l'intéressent puis de lancer les calculs des modes élémentaires de flux du réseau d'intérêt.\\ 
Cette interface est disponible en trois langues (Français, Anglais, Allemand), via des icône sur le coté droit du menu accessible à tout moment sur chacune des pages .

\subsubsection{Création d'un nouveau réseau}
Si l'utilisateur clique sur le bouton de création d'un nouveau réseau métabolique, une première page Web s'affiche. Il peut alors initialiser les fichiers présents sur son disque dur et écrire les réactions qui composent son réseau. Il a aussi la possibilité de modifier une réaction (métabolites, coefficients stœchiométriques). Lorsqu'il ajoute ou modifie une réaction, les données sont récupérées afin de générer automatiquement les fichiers nécessaires au fonctionnement du logiciel.
L'utilisateur peut aussi générer un fichier \emph{DAT} s'il le désire.\\
Lorsqu'il choisi de passer à l'étape suivante, il peut créer les règles des gènes qui seront utilisées par \emph{regEfmtool} pour calculer les modes élémentaires.

\subsubsection{Réglage des paramètres de lancement}
Il est possible pour un utilisateur confirmé ou habitué à l'interface d'avoir accès à un mode de réglage avancé des paramètres de lancement, s'il le désire. Ces derniers sont fixés à des valeurs par défaut pour les débutants. 

\subsubsection{Chargement d'un réseau préexistant}
Si l'utilisateur clique sur le bouton de chargement d'un réseau depuis la page d'accueil, il doit charger une série de fichiers (depuis le disque dur de son ordinateur) nécessaires au bon fonctionnement du logiciel. 

\subsubsection{Lancement du programme}
Lorsque l'utilisateur a créé son réseau manuellement, il doit ensuite choisir les paramètres de calcul de \textit{regEfmtool}. Nous avons fait le choix d'utiliser des cases à cocher en fonction de ce qu'il choisi. Pour les utilisateurs non expérimentés, les choix de base seront pré-sélectionnés. Il suffira ensuite de cliquer sur le bouton \emph{Lancement} pour avoir les résultats générés par le logiciel.

\subsubsection{Résultats}
Enfin, la visualisation des résultats apparaît de façon claire et conviviale à l'utilisateur au travers de l'interface Web. De plus, les résultats sont également présents dans des fichiers pour une analyse ultérieure. %Par ailleurs, nous allons essayer de mettre en place un dispositif d'annotations des fichiers.

\subsubsection{Aide en ligne}

Des pages d'aide sont aussi consultables pour la création d'un nouveau réseau.
Sur chaque page une lien hypertexte fait apparaître une page d'aide indiquant à l'utilisateur le fonctionnement de la page où il se trouve.\\
Ces pages d'aide sont également disponibles en trois langues (Français, Anglais, Allemand).
		
\subsection{Besoins non fonctionnels}

\subsubsection{Portabilité}
L'utilisation de \textit{regEfmtool} s'appuie sur d'autres logiciels, nécessitant par exemple la version 1.7 de Java. De ce fait, ils sont libres d'utilisation pour le secteur académique. L'interface est livrée avec tous les fichiers de configuration (pré-existants ou nouvellement créés) et indépendante du système d'exploitation. 

\subsubsection{Documentation}
L'écriture du code est constituée de commentaires qui permettent la maintenance du code ainsi qu'une éventuelle amélioration de ce dernier par un tiers.\\
L'interface Web créée, quand à elle, est fournie avec une documentation sur son installation et son utilisation disponible directement sur la page d’accueil du site..
