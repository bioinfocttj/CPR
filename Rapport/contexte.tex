%%%%%%%%%%%%%%%%%%%%%%%%%%%%%%
\section{Contexte}
%%%%%%%%%%%%%%%%%%%%%%%%%%%%%%

\subsection{Sujet}
Le métabolisme d'une cellule est un système complexe de transformations moléculaires et énergétiques qui se déroulent de manière ininterrompue dans la cellule et mettant en jeu un ensemble de réactions dites métaboliques. 
Ces réactions impliquent différents types de métabolites qui, suivant leurs positions dans la réaction, sont appelés substrats ou produits et catalysés par des enzymes.\\
La représentation visuelle des grandes fonctions métaboliques (glycolyse ou photosynthèse par exemple) sous la forme de réseaux pourrait être un outil précieux pour faciliter l'avancement des travaux des chercheurs. Cette modélisation de réactions permet de réaliser des requêtes complexes comme, par exemple, le calcul (et la prédiction) de tous les métabolites pouvant être générés à partir d'un ensemble de composés sources.\\
Quelques logiciels disponibles internationalement permettent de travailler et d'automatiser l'étude de ces réseaux, tout en proposant des fonctionnalités telles que le calcul des modes élémentaires de flux ou la recherche de \textit{minimal cut sets}\footnote{Un \textit{minimal cut sets}~\cite{mcs:url} (MCS) est un ensemble minimal (irréductible)de réactions dans le réseau dont l'activation va certainement conduire à une défaillance de certaines fonctions du réseau.}. Cependant, ils peuvent être dépendants de logiciels non libres comme MATLAB ou alors ne pas avoir d'interface utilisateur conviviale, ce qui est le cas avec  \textit{regEfmtool} (logiciel sur lequel nous nous appuierons pour ce projet).

\subsection{Objectif}
Dans le cadre de cette U.E., il nous était demandé de créer un programme (package et documentation à fournir à la fin du projet) nous permettant d'approfondir ou d'apprendre un langage de programmation, et de réaliser une analyse critique du travail effectué. L'objectif de ce projet était donc de réaliser une interface graphique du programme \textit{regEfmtool} qui ne pouvait être utilisé jusqu'alors que par des commandes à l'aide d'une console. \\

Deux possibilités s'offraient à nous pour réaliser cette interface: nous avions le choix entre les technologies Web et le langage Java. Nous avons choisi de nous appuyer sur les technologies Web. En effet, beaucoup de programmes utilisent une interface de type site Web leur permettant d'une part de créer une interface graphique conviviale, grâce aux feuilles de style, tout en permettant une certaine modularité. 